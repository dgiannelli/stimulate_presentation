\documentclass[10pt,t,xcolor=dvipsnames,aspectratio=169]{beamer}

\usetheme{Custom}

\usepackage{sansmathfonts}

%\usepackage[italian]{babel}
\usepackage[T1]{fontenc}
\usepackage[utf8]{inputenc}

\usepackage{mathtools}

\usepackage{tikz}
\usetikzlibrary{decorations,decorations.markings}
\usetikzlibrary{decorations.shapes}
\usepackage{import} % inputs with relative path
\let\pgfimageWithoutPath\pgfimage 
\renewcommand{\pgfimage}[2][]{\pgfimageWithoutPath[#1]{gfx/#2}}


\title[Fellow project presentation]{Hybrid quantum-classical simulations in lattice gauge theories}
\author{Giovanni Iannelli}
%\supervisor{}
\institute{Humboldt University of Berlin}
\date{15 January 2019}

\newlength\leftsidebar
\makeatletter
\setlength\leftsidebar{\beamer@leftsidebar}
\makeatother

\begin{document}

\hoffset=-0.5\leftsidebar
{
    \usebackgroundtemplate{
        \put(320,-230){%
        \tikz\node[opacity=0.2] {\pgfuseimage{humboldt}};
        }
    }
\begin{frame}[plain,t]
\titlepage
\end{frame}
}
\hoffset=0in % restore left margin

%\section*{}
%\begin{frame}[c]
%  \frametitle{Sommario}
%  \tableofcontents[hidesubsections]
%\end{frame}

\section{Introduction}

\tikzset{->-/.style={decoration={markings,mark=at position .55 with {\arrow{to}}},postaction={decorate}}} 
\tikzset{
    partial ellipse/.style args={#1:#2:#3}{
        insert path={+ (#1:#3) arc (#1:#2:#3)}
    }
}

\begin{frame}
    \frametitle{Quantum Computer and Quantum Simulation}
    \begin{itemize}
        \item
            Quantum computers memory is made by $n$ connected qubits.
        \item
            A basis of one qubit state space is $\{|0>,|1>\}$.
        \item
            The dimension of the $n$ qubits state space is $2^n$.
        \item
            The QPU (Quantum Processing Unit):
            \begin{itemize}
                \item Applies unitary transformations on the qubits.
                \item Measures the qubits values.
            \end{itemize}
        \item
            Quantum simulation:
            \begin{itemize}
                \item
                    Mapping a physical system into a qubits system.
                \item
                    Creating a qubits state and measuring observables with a quantum computer.
            \end{itemize}
    \end{itemize}
\end{frame}

\section{Objective}

\begin{frame}
    \frametitle{Looking for the ground state of a qubits system}
    \begin{itemize}
        \item
            We are developing a hybrid variational method for finding the ground state with a given Hamiltonian:
            \begin{itemize}
                \item Selects a qubits quantum state.
                \item Measures the energy with a quantum computer.
                \item Iterates on a new quantum state until the minimum energy is found.
            \end{itemize}
        \item
            Difficulties:
            \begin{itemize}
                \item Quantum measures are probabilistic.
                \item Quantum decoherence.
                \item Many parameters.
            \end{itemize}
        \item
            We need a stochastic, robust and scalable algorithm that can deal with noise:
            \begin{itemize}
                \item It has to work with random energy values $\Rightarrow$ error afflicted.
                \item It has to deal with impurities of quantum states.
                \item It has to work with many qubits states of future quantum architectures.
            \end{itemize}
    \end{itemize}
\end{frame}

\section{Aim}

\begin{frame}
    \frametitle{Application to statistical physics and lattice gauge theories}
    \begin{itemize}
        \item
            We will start with 1+1 dimensional models that are of interest in condensed matter and high energy physics:
            \begin{itemize}
                \item Heisenberg model.
                \item Schwinger model.
                \item $\mathbb CP^{N-1}$ model.
                \item Gross-Neveu model.
            \end{itemize}
        \item
            In presence of a $\theta$ term, MC simulations of gauge theories suffer from the sign problem (complex action),
            while quantum simulators don't (no integral evaluation, only direct measures).
        \item
            With Quantum Computers it may be possible to have results not obtainable with classic computers.
        \item
            We plan to apply the same idea to 2+1 dimensional gauge theories, abelian and non-abelian, and coupled to matter.
    \end{itemize}
\end{frame}

%\section{Methodology}
%\begin{frame}
%    \begin{itemize}
%        \item
%            We are working with 
%    \end{itemize}
%\end{frame}

\section{Expected results and impact}

\begin{frame}
    \frametitle{Quantum advantage}
    \begin{itemize}
        \item
            There is a Quantum advantage if a quantum computer can solve a problem that a classical computer cannot.
        \item
            Not achieved yet. IBM Q, Rigetti, Google and NASA are working on it for years.
        \item
            We expect that quantum physics problems will have greater advantages from quantum computers:
            \begin{itemize}
                \item Quantum systems are easier to map to qubits states. They share QM rules.
                \item Integrating path integrals is very resource demanding.
            \end{itemize}
        \item
            System with the sign problem could show quantum advantages:
            \begin{itemize}
                \item Path integrals evaluation becomes very complicated.
                \item In some cases impossible.
                \item Even an imprecise result with a quantum computer could show a quantum advantage.
            \end{itemize}
    \end{itemize}
\end{frame}

\hoffset=-0.5\leftsidebar
{
    \usebackgroundtemplate{
        \put(320,-230){%
        \tikz\node[opacity=0.2] {\pgfuseimage{humboldt}};
        }
    }
\begin{frame}[plain,t]
\titlepage
\end{frame}
}
\hoffset=0in % restore left margin

\section{Appendix}

\begin{frame}
    \frametitle{The sign problem in computational physics}
    \begin{itemize}
        \item
            The sign problem arises when evaluating highly oscillatory path integrals.
        \item
            It's one of the main problems in computational physics.
            It affects MC simulations of:
            \begin{itemize}
                \item Many electrons systems at low temperatures.
                \item Nuclei and neutron stars.
                \item Quark matter and vacuum in QCD.
            \end{itemize}
        \item
            Quantum simulators don't suffer from it.
            They don't need numerical integration of path integrals.
            Measures are performed directly on quantum states.
        \item
            Quantum computers may provide a complementary numerical method in these fields.
    \end{itemize}
\end{frame}

\end{document}
