\documentclass[10pt,t,xcolor=dvipsnames,aspectratio=169]{beamer}

\usetheme{Custom}

\usepackage{sansmathfonts}

%\usepackage[italian]{babel}
\usepackage[T1]{fontenc}
\usepackage[utf8]{inputenc}

\usepackage{mathtools}

\usepackage{tikz}
\usetikzlibrary{decorations,decorations.markings}
\usetikzlibrary{decorations.shapes}
\usepackage{import} % inputs with relative path
\let\pgfimageWithoutPath\pgfimage 
\renewcommand{\pgfimage}[2][]{\pgfimageWithoutPath[#1]{gfx/#2}}


\title[Fellow project]{Hybrid quantum-classical simulations in lattice gauge theories}
\author{Giovanni Iannelli}
%\supervisor{Prof. Massimo D'Elia}
\institute{Humboldt University of Berlin}
\date{15 January 2019}

\newlength\leftsidebar
\makeatletter
\setlength\leftsidebar{\beamer@leftsidebar}
\makeatother

\begin{document}

\hoffset=-0.5\leftsidebar
\begin{frame}[plain,t]
\titlepage
\end{frame}
\hoffset=0in % restore left margin

%\section*{}
%\begin{frame}[c]
%  \frametitle{Sommario}
%  \tableofcontents[hidesubsections]
%\end{frame}

%\section{Introduzione}

\tikzset{->-/.style={decoration={markings,mark=at position .55 with {\arrow{to}}},postaction={decorate}}} 
\tikzset{
    partial ellipse/.style args={#1:#2:#3}{
        insert path={+ (#1:#3) arc (#1:#2:#3)}
    }
}

\begin{frame}
    \frametitle{Quantum computer and quantum simulation}
    \begin{itemize}
        \item
            The QC memory is made by connected qubits (basis: $\{|0>,|1>\}$)
        \item
            The QPU:
            \begin{itemize}
                \item Applies unitary transformations on the qubits
                \item Measures the qubits values
            \end{itemize}
        \item
            Quantum simulation:
            \begin{itemize}
                \item
                    Mapping a physical system into a qbits system
                \item
                    Measuring observables with a quantum computer
            \end{itemize}
    \end{itemize}
\end{frame}

\begin{frame}
    \frametitle{Looking for the ground state}
    \begin{itemize}
        \item
            We are developing a hybrid variational method:
            \begin{itemize}
                \item Selects a qubit quantum state
                \item Measures the energy with a quantum computer
                \item Iterates on a new quantum state until the minimum energy is found
            \end{itemize}
        \item
            Difficulties:
            \begin{itemize}
                \item Quantum measures are probabilistic
                \item Quantum decoherence
            \end{itemize}
        \item
            We need a robust and scalable algorithm
    \end{itemize}
\end{frame}

\begin{frame}
    \frametitle{Application to lattice gauge theories}
    \begin{itemize}
        \item
            We will apply this method to the Schwinger model (QED in $1+1$ dimensions)
        \item
            In presence of a chemical potential or a $\theta$ term, the action is complex: sign problem in MC
        \item
            Quantum simulators don't suffer from the sign problem
        \item
            We plan to apply the same idea in higher dimensions
    \end{itemize}
\end{frame}

\begin{frame}
    \frametitle{The sign problem}
    \begin{itemize}
        \item
            The sign problem arises when integrating highly oscillatory functions
        \item
            It's one of the main problems in computational physics.
            It affects MC simulations of:
            \begin{itemize}
                \item Many electrons systems at low temperatures
                \item Nuclei and neutron stars
                \item Quark matter and vacuum in QCD
            \end{itemize}
        \item
            Quantum simulators don't suffer from it.
            They don't need numerical integration of path integrals.
            Measures are performed directly
    \end{itemize}
\end{frame}

\begin{frame}
    \frametitle{Quantum advantage}
    \begin{itemize}
        \item
            Using a quantum computer to solve a problem that a classical computer cannot
        \item
            Not achieved yet. Google and NASA (among others) are working on it for years.
        \item
            Quantum physics problems are more likely to have greater advanteges from quantum computers
        \item
            System with fermion sign problem could show quantum advantage
    \end{itemize}
\end{frame}

%\begin{frame}
%    \frametitle{Path integral}
%    \begin{columns}[T]
%        \begin{column}{0.6\textwidth}
%            \begin{itemize}
%                \item
%                    In QM e in QFT,
%                    quantità fisiche misurabili possono essere calcolate una volta noti i valori di aspettazione $\left<\hat{\mathcal O}[\phi]\right>$.
%                \item
%                    La formulazione del path integral fornisce un metodo per calcolare $\left<\hat{\mathcal O}[\phi]\right>$:
%                    è uguale al valor medio di $\mathcal O[\phi]$ 
%                    su tutte le possibili configurazioni (cammini) $\phi$,
%                    pesate secondo la funzione $e^{-S[\phi]}$.
%                \item
%                    $S[\phi]$ è l'azione Euclidea del sistema.
%                %\item
%                %    Soluzioni classiche corrispondono ai minimi di $S[\phi]$
%            \end{itemize}
%        \end{column}%
%        \begin{column}{0.4\textwidth}
%               \centering
%               \begin{tikzpicture}
%                   \clip (-1,-1) rectangle (2,4.5);
%                   \draw [thick,->-] (0,0) .. controls (0,6) and (-2,-3) .. (1,4);
%                   %\draw [->-] (0,0) to[bend left] (1,1) to[bend left] (2,0) to[bend left] (1,0) to (0,4);
%                   %\draw [->-] (0,0) .. controls (-0.5,2) .. (-1,3) .. controls (-2,2) .. (-0.5,2);
%                   \draw [thick,->-] (0,0) to (1,4);
%                   \draw [thick,->-] (0,0) to[bend right] (1,4);
%                   \node[anchor=north east] at (0,0) {$A$};
%                   \node[anchor=south west] at (1,4) {$B$};
%               \end{tikzpicture}
%        \end{column}
%    \end{columns}
%\end{frame}
%
%\begin{frame}
%    \frametitle{Topologie non banali}
%    \begin{columns}[T]
%        \begin{column}{0.6\textwidth}
%            \begin{itemize}
%                \item
%                    Sorgono difficoltà se lo spazio dei cammini $\phi$ è topologicamente non banale.
%                \item
%                    In tal caso il dominio di integrazione è diviso in classi di omotopia, caratterizzate da una carica topologica.
%                \item
%                    Due configurazioni sono omotope se è possibile ottenere una dall'altra con una deformazione continua.
%                %\item
%                %    Per ogni classe di omotopia vi è un minimo locale di $S[\phi]$,
%                %    e dunque una soluzione classica.
%                \item
%                    In generale, è necessario integrare su tutte le configurazioni in ogni classe di omotopia per ottenere risultati corretti.
%            \end{itemize}
%        \end{column}
%        \begin{column}{0.4\textwidth}
%            \centering
%            \begin{tikzpicture}
%                \draw[fill=RoyalBlue] (0,2.5) circle (0.75);
%                \draw[->-,thick] (0,3.5) arc (90:-90:1);
%                \draw[->-,thick] (0,1.5) arc (270:90:1);
%
%                \draw[fill=RoyalBlue] (0,0) circle (0.75);
%                \draw[>-,thick] (0.85,0) arc (0:-180:0.9);
%                \draw[>-,thick] (-0.95,0) arc (180:0:1);
%                \draw[>-,thick] (1.05,0) arc (0:-180:1.1);
%                \draw[>-,thick] (-1.15,0) arc (180:0:1);
%            \end{tikzpicture}
%        \end{column}
%    \end{columns}
%\end{frame}
%
%\begin{frame}
%    \frametitle{Topologia in QCD}
%    \begin{itemize}
%        \item
%            Le configurazioni dei campi di gauge in QCD sono topologicamente non banali,
%            e sono identificate da una carica topologica $Q \in \mathbb Z$.
%        %\item
%        %    Le corrispondenti soluzioni classiche sono chiamate istantoni.
%        \item
%            Ci sono anche quantità fisiche direttamente legate alla carica topologica,
%            come la massa del mesone $\eta'$ e la massa dell'assione,
%            che è una particella ipotetica candidata ad essere parte della materia oscura.
%        \item
%            La massa dell'assione è proporzionale alla suscettività topologica:
%            \[
%                \chi = \frac{\left<Q^2\right>}{V}
%            \]
%        \item
%            Una stima accurata e precisa della suscettività topologica in QCD su reticolo potrebbe essere molto utile per la scoperta sperimentale dell'assione.
%    \end{itemize}
%\end{frame}
%
%\begin{frame}
%    \frametitle{Integrazione con Monte Carlo}
%    \begin{columns}[T]
%        \centering
%        \begin{column}{0.64\textwidth}
%            \begin{itemize}
%                \item
%                    Il dominio di integrazione viene discretizzato e campionato con catene di Markov con distribuzione di equilibrio $\propto e^{-S}$.
%                \item
%                    Viene aggiornata una variabile alla volta.
%                    Nel limite al continuo, aggiornamenti locali corrispondono a trasformazioni di omotopia.
%                    $\Rightarrow$ non si può passare da una classe di omotopia a un'altra.
%                \item
%                    Nello spazio discretizzato un tunneling è ancora possibile,
%                    ma la probabilità decresce esponenzialmente avvicinandosi al continuo (freezing topologico).
%                \item
%                    Vicino al limite continuo, non è possibile effettuare misure corrette.
%            \end{itemize}
%        \end{column}
%        \begin{column}{0.34\textwidth}
%            \begin{center}
%            \begin{overprint}
%            \begin{tikzpicture}
%                \clip (-1,0) rectangle (2,5.4);
%                %\draw[thick,->-] (0,0) to (0.5,1);
%
%                \onslide<1>{\draw[thick,->-] (0.5,1) to (-0.5,2);}
%                \onslide<2->{\draw[thick,->-] (0.5,1) to (0.5,2);}
%
%                \onslide<1>{\draw[thick,->-] (-0.5,2) to (0,3);}
%                \onslide<2-3>{\draw[thick,->-] (0.5,2) to (0,3);}
%                \onslide<4>{\draw[thick,->-] (0.5,2) to (1,3);}
%
%                \onslide<-2>{\draw[thick,->-] (0,3) to (1,4);}
%                \onslide<3>{\draw[thick,->-] (0,3) to (0.5,4);}
%                \onslide<4>{\draw[thick,->-] (1,3) to (0.5,4);}
%
%                \onslide<-2>{\draw[thick,->-] (1,4) to (0.5,5);}
%                \onslide<3->{\draw[thick,->-] (0.5,4) to (0.5,5);}
%
%
%                %\draw[fill=RoyalBlue] (-0.1,-0.1) rectangle (0.1,0.1);
%
%                \node[anchor=north east] at (0.5,1) {$A$};
%                \draw[fill=RoyalBlue] (0.4,0.9) rectangle (0.6,1.1);
%
%                \onslide<1>{\draw[fill=RoyalBlue] (-0.6,1.9) rectangle (-0.4,2.1);}
%                \onslide<2->{\draw[fill=RoyalBlue] (0.4,1.9) rectangle (0.6,2.1);}
%
%                \onslide<-3>{\draw[fill=RoyalBlue] (-0.1,2.9) rectangle (0.1,3.1);}
%                \onslide<4>{\draw[fill=RoyalBlue] (0.9,2.9) rectangle (1.1,3.1);}
%
%                \onslide<-2>{\draw[fill=RoyalBlue] (0.9,3.9) rectangle (1.1,4.1);}
%                \onslide<3->{\draw[fill=RoyalBlue] (0.4,3.9) rectangle (0.6,4.1);}
%
%                \draw[fill=RoyalBlue] (0.4,4.9) rectangle (0.6,5.1);
%                \node[anchor=south west] at (0.5,5) {$B$};
%            \end{tikzpicture}
%            \end{overprint}
%            \end{center}
%        \end{column}
%    \end{columns}
%\end{frame}
%
%\begin{frame}
%    \frametitle{Critical slowing down e algoritmi a cluster}
%    \begin{itemize}
%        \item
%            Un fenomeno simile si manifesta in sistemi termodinamici vicino alla temperatura critica.
%        \item
%            Il tempo di rilassamento diverge con legge di potenza alla temperatura critica
%            (il freezing topologico è invece esponenziale).
%        \item
%            In alcuni sistemi (i.e. sistemi di spin)
%            il problema è risolto con l'implementazione di algoritmi a cluster.
%        \item
%            Usando le simmetrie del sistema, è possibile aggiornare simultaneamente un numero grande di variabili (cluster)
%            in accordo con la distribuzione di Boltzmann (oppure $e^{-S}$).
%    \end{itemize}
%\end{frame}
%
%\begin{frame}
%    \frametitle{Algoritmi a cluster per contrastare il freezing topologico? I}
%    \begin{itemize}
%        \item
%            Non è ancora disponibile un algoritmo a cluster per la QCD.
%            Vi sono soltanto diversi metodi che aiutano a contrastare il freezing.
%        \item
%            Molte proprietà della carica topologica possono comunque essere studiate in modelli più semplici (modelli giocattolo).
%        \item
%            Particella quantistica sul cerchio: il problema si risolve \emph{tagliando} e \emph{ricucendo} i cammini con diversi avvolgimenti.
%    \end{itemize}
%\end{frame}
%
%\begin{frame}
%    \frametitle{Algoritmi a cluster per contrastare il freezing topologico? II}
%    \begin{itemize}
%        \item
%            Lo scopo di questa tesi è stato di estendere questa idea a una teoria di gauge:
%            la teoria $U(1)$ di pura gauge in due dimensioni.
%        \item
%            In questo modo si possono ottenere soluzioni esatte in un modello molto più simile alla QCD rispetto alla particella sul cerchio.
%            In particolare:
%            \begin{itemize}
%                \item
%                    È possibile implementare in questo modello le tecniche usate in QCD per valutarne l'efficacia.
%                \item
%                    È possibile studiare il comportamento della carica topologica in una teoria di gauge vicino al limite continuo.
%            \end{itemize}
%    \end{itemize}
%\end{frame}
%
%
%\section{Modello giocattolo}
%
%\begin{frame}
%    \frametitle{Teoria nel continuo}
%    \begin{itemize}
%        \item
%            La Lagrangiana Euclidea del sistema è:
%            \[
%                \mathcal L \equiv \frac{1}{4}F_{\mu\nu}F_{\mu\nu} \qquad \text{dove } F_{\mu\nu} \equiv \partial_\mu A_\nu - \partial_\nu A_\mu
%            \]
%            %con $F_{\mu\nu} \equiv \partial_\mu A_\nu - \partial_\nu A_\mu$
%        \item
%            La trasformazione di gauge locale dei campi indotta da $G(x)\equiv e^{i\theta(x)}\in U(1)$ è:
%            \[
%                A_\mu(x) = A_\mu(x) - \frac{1}{g}\theta(x)
%            \]
%        \item
%            Il valore di aspettazione di un operatore $\widehat{\mathcal O}[A]$ è:
%            \[
%                \left<\widehat{\mathcal O}[A]\right> = \frac{\int[\mathrm dA]\, \mathcal O[A] e^{-S[A]}}{\int[\mathrm dA]\,e^{-S[A]}}
%            \]
%            dove l'azione è:
%            \[
%                S[A] \equiv \int\mathrm d^2x\,\mathcal L[A(x)]
%            \]
%    \end{itemize}
%\end{frame}
%
%\begin{frame}
%    \frametitle{Carica topologica nel continuo}
%    \begin{itemize}
%        \item
%            La densità di carica topologica è:
%            \[
%                q(x) \equiv \frac{g}{4\pi}\epsilon_{\mu\nu}F_{\mu\nu}(x)
%            \]
%        \item
%            La carica topologica è:
%            \[
%                Q \equiv \int\mathrm d^2x\,q(x)
%            \]
%        \item
%            $Q$ assume solo valori interi se lo spazio è:
%            \begin{itemize}
%                \item
%                    Infinito con azione finita $\Rightarrow$ con campi asintoticamente nulli.
%                \item
%                    Una varietà compatta orientabile.
%            \end{itemize}
%    \end{itemize}
%\end{frame}
%
%\begin{frame}
%    \frametitle{Teoria su reticolo: siti e link}
%    \begin{columns}[T]
%        \centering
%        \begin{column}{0.59\textwidth}
%            \begin{itemize}
%                \item
%                    La teoria viene discretizzata su una griglia finita di $N\times N$ siti indicati con $s$.
%                \item
%                    Le condizioni al bordo sono periodiche.
%                \item
%                    L'unità di misura delle lunghezze è il passo reticolare $a$.
%                \item
%                    I campi di gauge sono incapsulati nelle linee di Schwinger sui link:
%                    \[
%                        U_\mu(s) \equiv e^{iga\left[A_\mu(s)+\mathcal O(a)\right]} = U^*_{-\mu}(s+\hat\mu)
%                    \]
%                dove $U_\mu(s) \in U(1)$ è il link di lunghezza $a$ che parte da $s$ ed è diretto in direzione $\mu$.
%            \end{itemize}
%        \end{column}
%        \begin{column}{0.39\textwidth}
%            \centering
%			\begin{tikzpicture}[x=1cm,y=1cm]
%				\draw[step=1, help lines, dashed, color=black!30] (0,0) grid (2.5,4.5);
%				\draw[->,thick,black] (0,0) -- (2.5,0) node [right] {$x_1$};
%				\draw[->,thick,black] (0,0) -- (0,4.5) node [above] {$x_2$};
%
%				\draw[->-,very thick,RoyalBlue] (1,1) -- node [black,yshift=0.4cm] {$U_1(1,1)$} (2,1);
%				\draw[->-,very thick,RoyalBlue] (1,4) -- node [black,right] {$U_{-2}(1,4)$} (1,3);
%			\end{tikzpicture}
%        \end{column}
%    \end{columns}
%\end{frame}
%
%\begin{frame}
%    \frametitle{Teoria su reticolo: trasformazioni di gauge locali}
%    \centering
%    \begin{tikzpicture}[x=1.5cm,y=1.5cm]
%        \draw[step=1, help lines, dashed, color=black!30] (0,0) grid (2.5,2.5);
%        \draw[->,thick,black] (0,0) -- (2.5,0) node [right] {$x_1$};
%        \draw[->,thick,black] (0,0) -- (0,2.5) node [above] {$x_2$};
%
%        \draw[->-,very thick,RoyalBlue] (1,1) -- node [black,yshift=0.3cm] {$U_1(s)$} (2,1);
%        \draw[->-,very thick,RoyalBlue] (1,1) -- node [black,left] {$U_2(s)$} (1,2);
%        \draw[->-,very thick,RoyalBlue] (1,1) -- node [black,yshift=-0.3cm] {$U_{-1}(s)$} (0,1);
%        \draw[->-,very thick,RoyalBlue] (1,1) -- node [black,right] {$U_{-2}(s)$} (1,0);
%
%        \draw[->,thick,black] (3,1) -- node [black,yshift=0.4cm] {$G\in U(1)$} (4,1);
%
%        \draw[step=1, help lines, dashed, color=black!30, xshift=0.75cm] (4,0) grid (6.5,2.5);
%        \draw[->,thick,black] (4.5,0) -- (7,0) node [right] {$x_1$};
%        \draw[->,thick,black] (4.5,0) -- (4.5,2.5) node [above] {$x_2$};
%
%        \draw[->-,very thick,RoyalBlue] (5.5,1) -- node [black,yshift=0.3cm] {$GU_1(s)$} (6.5,1);
%        \draw[->-,very thick,RoyalBlue] (5.5,1) -- node [black,left] {$GU_2(s)$} (5.5,2);
%        \draw[->-,very thick,RoyalBlue] (5.5,1) -- node [black,yshift=-0.3cm] {$GU_{-1}(s)$} (4.5,1);
%        \draw[->-,very thick,RoyalBlue] (5.5,1) -- node [black,right] {$GU_{-2}(s)$} (5.5,0);
%    \end{tikzpicture}
%\end{frame}
%
%\begin{frame}
%    \frametitle{Teoria su reticolo: oggetti gauge invarianti e plaquette}
%    \begin{columns}[T]
%        \centering
%        \begin{column}{0.59\textwidth}
%            \begin{itemize}
%                \item 
%                    Ogni linea di Schwinger valutata su un percorso chiuso è un invariante di gauge.
%                \item
%                    La plaquette è la linea di Schwinger attorno a un quadrato di lato $a$.
%                    È data dal prodotto dei suoi link.
%                \item
%                    Nella plaquette è contenuta l'informazione del tensore dei campi di gauge $F_{\mu\nu}$:
%                    \[
%                        \mathcal P_{\mu\nu}(s) = e^{iga^2\left[F_{\mu\nu}(s)+\mathcal O(a)\right]}
%                    \]
%            \end{itemize}
%		\end{column}
%		\begin{column}{0.39\textwidth}
%			\begin{tikzpicture}[x=0.8cm,y=0.8cm]
%				\draw[step=1, help lines, dashed, color=black!30] (0,0) grid (3.5,4.5);
%				\draw[->,thick,black] (0,0) -- (3.5,0) node [right] {$x_1$};
%				\draw[->,thick,black] (0,0) -- (0,4.5) node [above] {$x_2$};
%
%				\draw[->-,very thick,RoyalBlue] (2,2) -- node [black,yshift=-0.3cm] {\scriptsize $U_1(s)$} (3,2);
%				\draw[->-,very thick,RoyalBlue] (3,2) -- node [black,right] {\scriptsize $U_{2}(s+\hat1)$} (3,3);
%				\draw[->-,very thick,RoyalBlue] (3,3) -- node [black,yshift=0.3cm] {\scriptsize $U_{-1}(s+\hat1+\hat2)$} (2,3);
%				\draw[->-,very thick,RoyalBlue] (2,3) -- node [black,left] {\scriptsize $U_{-2}(s+\hat2)$} (2,2);
%			\end{tikzpicture}
%
%		\end{column}
%	\end{columns}
%\end{frame}
%
%\begin{frame}
%    \frametitle{Azione su reticolo}
%    \begin{itemize}
%        \item
%            L'azione su reticolo può essere definita in termini delle plaquette (gauge invarianti):
%            \[
%                S[U] \equiv \beta\sum_s\left(1-\Re\,\mathcal P_{12}(s)\right)
%            \]
%            Usando $a^2\sum_s \xrightarrow{a\to0} \int\mathrm d^2x$, converge all'azione continua:
%            \[
%                S = \beta\sum_s\frac{1}{4}g^2a^4F_{\mu\nu}(s)F_{\mu\nu}(s) + \mathcal O\left(a^5\right)
%            \xrightarrow{a\to0} \beta g^2a^2\int\mathrm d^2x\,\frac{1}{4}F_{\mu\nu}(x)F_{\mu\nu}(x)
%            \]
%            se il parametro $\beta$ è posto uguale a:
%            \[
%                \beta \equiv \frac{1}{g^2a^2}
%            \]
%    \end{itemize}
%\end{frame}
%
%\begin{frame}
%    \frametitle{Carica topologica su reticolo}
%    \begin{itemize}
%        \item
%            Si può dare anche una definizione gauge invariante della carica topologica in termini delle plaquette:
%            \[
%                Q \equiv \frac{1}{2\pi}\sum_s\arg \mathcal P_{12}(s)
%            \]
%            che è la somma dei contributi locali:
%            \[
%                q(s) \equiv \frac{1}{2\pi}\arg \mathcal P_{12}(s)
%            \]
%        \item
%            Invertendo il verso di percorrenza delle plaquette, la carica topologica cambia il segno.
%        \item
%            Con condizioni al bordo periodiche,
%            questa definizione della carica topologica assume sempre valori interi.
%            Come nella teoria continua.
%    \end{itemize}
%\end{frame}
%
%\begin{frame}
%    \frametitle{Misura dei valori di aspettazione con un Monte Carlo}
%    \begin{itemize}
%        \item
%            Il valore di aspettazione di un operatore $\hat{\mathcal O}[U]$ definito su reticolo è:
%            \[
%                \left<\hat{\mathcal O}[U]\right> \equiv \frac{\int[\mathrm dU]\, \mathcal O[U] e^{-S[U]}}{\int[\mathrm dU]\,e^{-S[U]}}
%            \]
%            dove le variabili dinamiche sono i link.
%        \item
%            Una catena di Markov di configurazioni dei link viene generata in accordo con la distribuzione $\propto e^{-S[U]}$.
%        \item
%            La media di $\mathcal O[U]$ valutata sulla catena di Markov converge al valore di aspettazione $\left<\hat{\mathcal O}[U]\right>$
%            (teorema ergodico).
%    \end{itemize}
%\end{frame}
%
%\begin{frame}
%    \frametitle{Estrapolazione del limite al continuo}
%    \begin{itemize}
%        \item
%            Vengono effettuate diverse misure di $\left<\hat{\mathcal O}[U]\right>$ con valori di $a$ decrescente
%            a volume fisico $L^2=N^2a^2$ costante.
%        \item
%            Con un fit dell'andamento ad $a$ decrescente si può stimare il valore limite in $a=0$.
%        \item
%            In questa teoria, l'accoppiamento $g$ non rinormalizza, e la linea di volume costante è data da:
%            \[
%                \frac{\beta}{N^2} = \text{costante}
%            \]
%        \item
%            In seguito verranno riportate misure dell'energia di plaquette media $E[\mathcal P]$ e della suscettività topologica $\chi/g^2$
%            sulla linea a volume costante $\beta/N^2=1/80$.
%            \[
%                E[\mathcal P] \equiv \frac{\beta}{N^2}\sum_s(1-\Re\,\mathcal P_{12}(s)) \qquad \chi/g^2 \equiv \beta\frac{Q^2}{N^2}
%            \]
%    \end{itemize}
%\end{frame}
%
%
%\section{Algoritmo locale}
%
%\begin{frame}
%    \frametitle{Metropolis-Hastings locale I}
%    \begin{columns}[T]
%        \centering
%        \begin{column}{0.59\textwidth}
%            \begin{itemize}
%                \item
%                    Solo un link $U$ alla volta viene aggiornato con un M-H,
%                    che è un algoritmo accept/reject che consente di campionare valori da una generica PDF.
%                \item
%                    La PDF dei link U dipende dall'azione:
%                    \[
%                        p(U) \propto e^{-S} \qquad S = \beta\sum_s(1-\Re\,\mathcal P_{12})
%                    \]
%                \item
%                    Solo i valori delle plaquette $\mathcal P_1$ e $\mathcal P_2$ cambiano aggiornando $U$.
%                \item
%                    La PDF di $U$ dipende solo dai link contenuti nelle staples $\mathcal S_1$ e $\mathcal S_2$.
%                    Tutti gli altri possono essere marginalizzati fuori dalla PDF.
%            \end{itemize}
%        \end{column}
%        \begin{column}{0.39\textwidth}
%            \centering
%            \begin{tikzpicture}[x=1.4cm,y=1.4cm]
%                \draw[step=1, help lines, dashed, color=black!30] (0,0) grid (2.5,3.5);
%                \draw[->,thick,black] (0,0) -- (2.5,0) node [right] {$x_1$};
%                \draw[->,thick,black] (0,0) -- (0,3.5) node [above] {$x_2$};
%
%                \draw[->-,very thick,orange] (1,2) node [black,left] {$U$} --  (2,2);
%
%
%                %top staple
%                \draw[->-,very thick,RoyalBlue] (2,2) -- (2,3);
%                \draw[->-,very thick,RoyalBlue] (2,3) -- node [black,above] {$\mathcal S_1$} (1,3);
%                \draw[->-,very thick,RoyalBlue] (1,3) -- (1,2);
%
%                %bot staple
%                \draw[->-,very thick,RoyalBlue] (2,2) -- (2,1);
%                \draw[->-,very thick,RoyalBlue] (2,1) -- node [black,below] {$\mathcal S_2$} (1,1);
%                \draw[->-,very thick,RoyalBlue] (1,1) -- (1,2);
%
%                %top plaq
%                \draw[->-,thick,dashed,black] (1.2,2.2) -- (1.8,2.2);
%                \draw[->-,thick,dashed,black] (1.8,2.2) -- (1.8,2.8);
%                \draw[->-,thick,dashed,black] (1.8,2.8) -- (1.2,2.8);
%                \draw[->-,thick,dashed,black] (1.2,2.8) -- (1.2,2.2);
%                \node at (1.5,2.5) {$\mathcal P_1$};
%
%                %bot plaq
%                \draw[->-,thick,dashed,black] (1.2,1.2) -- (1.8,1.2);
%                \draw[->-,thick,dashed,black] (1.8,1.2) -- (1.8,1.8);
%                \draw[->-,thick,dashed,black] (1.8,1.8) -- (1.2,1.8);
%                \draw[->-,thick,dashed,black] (1.2,1.8) -- (1.2,1.2);
%                \node at (1.5,1.5) {$\mathcal P_2$};
%
%                %\node[anchor={north east}] at (1,1) {$\mathcal P_1$};
%                %\node[anchor={north east}] at (2,1) {$\mathcal P_2$};
%
%            \end{tikzpicture}
%        \end{column}
%    \end{columns}
%\end{frame}
%
%\begin{frame}
%    \frametitle{Metropolis-Hastings locale II}
%    \begin{columns}[T]
%        \centering
%        \begin{column}{0.59\textwidth}
%            \begin{itemize}
%                \item
%                    La PDF di $U$ risulta essere:
%                    \[
%                        p(U) \propto e^{k\cos x} \qquad
%                        \text{con }\begin{dcases}
%                            W \equiv \mathcal S_1 + \mathcal S_2 \\
%                            k \equiv \beta\left|W\right| \\
%                            x \equiv \arg(WU)
%                        \end{dcases}
%                    \]
%                \item
%                    La distribuzione risulta convergere a una Gaussiana nel limite al continuo $\beta\to0$ (Metodo di Laplace).
%                \item
%                    È stato sfruttato questo fatto per ottimizzare l'algoritmo in modo da ottenere accettanze vicine al 100\% per $\beta$ piccoli.
%            \end{itemize}
%        \end{column}
%        \begin{column}{0.39\textwidth}
%            \centering
%            \begin{tikzpicture}[x=1.4cm,y=1.4cm]
%                \draw[step=1, help lines, dashed, color=black!30] (0,0) grid (2.5,3.5);
%                \draw[->,thick,black] (0,0) -- (2.5,0) node [right] {$x_1$};
%                \draw[->,thick,black] (0,0) -- (0,3.5) node [above] {$x_2$};
%
%                \draw[->-,very thick,orange] (1,2) node [black,left] {$U$} --  (2,2);
%
%                %top staple
%                \draw[->-,very thick,RoyalBlue] (2,2) -- (2,3);
%                \draw[->-,very thick,RoyalBlue] (2,3) -- node [black,above] {$\mathcal S_1$} (1,3);
%                \draw[->-,very thick,RoyalBlue] (1,3) -- (1,2);
%
%                %bot staple
%                \draw[->-,very thick,RoyalBlue] (2,2) -- (2,1);
%                \draw[->-,very thick,RoyalBlue] (2,1) -- node [black,below] {$\mathcal S_2$} (1,1);
%                \draw[->-,very thick,RoyalBlue] (1,1) -- (1,2);
%
%                %top plaq
%                \draw[->-,thick,dashed,black] (1.2,2.2) -- (1.8,2.2);
%                \draw[->-,thick,dashed,black] (1.8,2.2) -- (1.8,2.8);
%                \draw[->-,thick,dashed,black] (1.8,2.8) -- (1.2,2.8);
%                \draw[->-,thick,dashed,black] (1.2,2.8) -- (1.2,2.2);
%                \node at (1.5,2.5) {$\mathcal P_1$};
%
%                %bot plaq
%                \draw[->-,thick,dashed,black] (1.2,1.2) -- (1.8,1.2);
%                \draw[->-,thick,dashed,black] (1.8,1.2) -- (1.8,1.8);
%                \draw[->-,thick,dashed,black] (1.8,1.8) -- (1.2,1.8);
%                \draw[->-,thick,dashed,black] (1.2,1.8) -- (1.2,1.2);
%                \node at (1.5,1.5) {$\mathcal P_2$};
%
%                %\node[anchor={north east}] at (1,1) {$\mathcal P_1$};
%                %\node[anchor={north east}] at (2,1) {$\mathcal P_2$};
%
%            \end{tikzpicture}
%        \end{column}
%    \end{columns}
%\end{frame}
%
%
%\begin{frame}
%    \frametitle{Accettanza dell'algoritmo locale: limite al continuo}
%    \centering
%    \import{gfx/}{local_acc.pgf}
%\end{frame}
%
%\begin{frame}
%    \frametitle{Algoritmo locale: simulazioni e verifica della correttezza}
%    \begin{itemize}
%        \item
%            Simulazioni per diversi parametri di $\beta$ e $N$ sulla linea $\beta/N^2=1/80$ sono state effettuate.
%        \item
%            Ogni aggiornamento di tutti i link in successione è considerato un'iterazione della catena di Markov.
%        \item
%            I valori medi sono stati stimati con i dati raccolti in $10^6$ iterazioni,
%            scartando i primi $2\times10^5$ per far termalizzare il sistema.
%        \item
%            Per verificare la correttezza del programma,
%            è stato confrontato il valore di $E[\mathcal P]$, ottenuto con $\beta=7.2$ e $N=24$, con quello riportato da D\"urr e Hoelbling nel 2005:
%            \[
%                \begin{aligned}
%                    E[\mathcal P] &= 0.52040(39) \quad \text{(D\"urr-Hoelbling)} \\
%                    E[\mathcal P] &= \input{tables/plaq_local.tex} \quad \text{(Algoritmo locale)}
%                \end{aligned}
%            \]
%    \end{itemize}
%\end{frame}
%
%\begin{frame}
%    \frametitle{Algoritmo locale: Limite al continuo dell'energia di plaquette media con BIAS}
%    \centering
%    \import{gfx/}{local_energy_cont.pgf}
%\end{frame}
%
%
%\begin{frame}
%    \frametitle{Algoritmo locale: storia della carica topologica}
%    \centering
%    \import{gfx/}{local_charge_history.pgf}
%\end{frame}
%
%\begin{frame}
%    \frametitle{Algoritmo locale: tempo di autocorrelazione della carica topologica}
%    \centering
%    \import{gfx/}{local_charge_corr.pgf}
%\end{frame}
%
%\section{Visualizzazione della carica topologica}
%
%\begin{frame}
%    \frametitle{Colormap della carica locale}
%    \begin{columns}[T]
%        \centering
%        \begin{column}{0.39\textwidth}
%            \begin{itemize}
%                \item
%                    La carica topologica locale può assumere valori in $(-1/2,1/2]$.
%                \item
%                    I singoli contributi possono essere rappresentati con una diverging colormap.
%                \item
%                    I valori positivi o negativi sono rappresentati con colori contrastanti a luminosità e saturazione variabile.
%                    I valori vicini a zero sono rappresentati in bianco.
%            \end{itemize}
%        \end{column}
%        \begin{column}{0.59\textwidth}
%            \begin{overprint}
%                \centerline{\import{gfx/}{raw_charge_cmap.pgf}}
%            \end{overprint}
%        \end{column}
%    \end{columns}
%\end{frame}
%
%\begin{frame}
%    \frametitle{Blur Gaussiano della carica}
%    \begin{columns}[T]
%        \centering
%        \begin{column}{0.39\textwidth}
%            \begin{itemize}
%                \item
%                    Su reticolo, la carica locale esibisce fluttuazioni UV di lunghezza d'onda $\sim a$.
%                \item
%                    Le fluttuazioni UV oscurano le fluttuazioni della carica continua, di lunghezza $\sim \rho$.
%                \item
%                    Occorre eseguire uno smoothing locale su distanze $\gg a$ ma $\ll \rho$.
%                \item
%                    È proposto un metodo per lo smoothing della carica che utilizza il blur Gaussiano.
%            \end{itemize}
%        \end{column}
%        \begin{column}{0.59\textwidth}
%            \begin{overprint}
%                \onslide<1>\centerline{\import{gfx/}{charge_blur1.pgf}}
%                \onslide<2>\centerline{\import{gfx/}{charge_blur2.pgf}}
%                \onslide<3>\centerline{\import{gfx/}{charge_blur3.pgf}}
%                \onslide<4>\centerline{\import{gfx/}{charge_blur4.pgf}}
%            \end{overprint}
%        \end{column}
%    \end{columns}
%\end{frame}
%
%\begin{frame}
%    \frametitle{Algoritmo locale: visualizzazione del freezing topologico}
%    \begin{columns}[T]
%        \centering
%        \begin{column}{0.39\textwidth}
%            \begin{itemize}
%                \item
%                    Sono stati scelti valori dei parametri per cui il topological freezing è totale.
%                \item
%                    Sono riportate le colormap della carica in quattro successive iterazioni dell'algoritmo locale.
%                \item
%                    I cluster si muovono e si deformano, ma il loro contributo rimane lo stesso, così come la carica totale.
%            \end{itemize}
%        \end{column}
%        \begin{column}{0.59\textwidth}
%            \begin{overprint}
%                \onslide<1>\centerline{\import{gfx/}{freezing1.pgf}}
%                \onslide<2>\centerline{\import{gfx/}{freezing2.pgf}}
%                \onslide<3>\centerline{\import{gfx/}{freezing3.pgf}}
%                \onslide<4>\centerline{\import{gfx/}{freezing4.pgf}}
%            \end{overprint}
%        \end{column}
%    \end{columns}
%\end{frame}
%
%\begin{frame}
%    \frametitle{Algoritmo locale: visualizzazione del tunneling topologico}
%    \begin{columns}[T]
%    \centering
%        \begin{column}{0.39\textwidth}
%            \begin{itemize}
%                \item
%                    Sono scelti ora parametri per cui il tunneling topologico è ancora possibile: avviene circa una volta ogni $500-1000$ iterazioni.
%                %\item
%                %    Un cluster blu scompare, e al suo posto compare un cluster rosso con contributo di segno opposto.
%                \item
%                    L'idea dell'algoritmo a cluster è di definire un algoritmo di Metropolis che possa ricreare artificialmente questo tunneling.
%            \end{itemize}
%        \end{column}
%        \begin{column}{0.59\textwidth}
%            \begin{overprint}
%                \onslide<1>\centerline{\import{gfx/}{local_inv1.pgf}}
%                \onslide<2>\centerline{\import{gfx/}{local_inv2.pgf}}
%                \onslide<3>\centerline{\import{gfx/}{local_inv3.pgf}}
%                \onslide<4>\centerline{\import{gfx/}{local_inv4.pgf}}
%            \end{overprint}
%        \end{column}
%    \end{columns}
%\end{frame}
%
%\section{Algoritmo a cluster}
%
%\begin{frame}
%    \frametitle{Inversione della topologia di un cluster}
%    \begin{columns}[T]
%        \centering
%        \begin{column}{0.59\textwidth}
%            \begin{itemize}
%                \item
%                    L'obiettivo è di definire un'operazione che consenta il tunnel topologico.
%                \item
%                    Coniugando ogni link in una plaquette, la carica cambia segno (è come invertire il percorso).
%                \item
%                    L'azione locale invece rimane invariata $\Rightarrow$ utile per Metropolis.
%                \item
%                    Coniugando ogni link in una regione cambia però l'azione delle plaquette esterne connesse al bordo.
%                \item
%                    Sono in generale troppe a cambiare. La mossa avrebbe un'accettanza troppo bassa, specialmente vicino al limite continuo.
%            \end{itemize}
%        \end{column}
%        \begin{column}{0.39\textwidth}
%            \begin{center}
%                \begin{overprint}
%                \begin{tikzpicture}[x=0.35cm,y=0.35cm]
%                    %internal links
%                    %\draw[step=1,color=RoyalBlue] (1,1) grid (3,3);
%
%                    \onslide<1>{\draw[fill=MidnightBlue] (2,2) rectangle (5,5);}
%                    \onslide<2>{\draw[fill=Maroon] (2,2) rectangle (5,5);}
%                    %\draw[step=1,white] (2,2) grid (5,5);
%
%                    %external
%                    \onslide<2>{\draw[fill=OliveGreen] (2,1) rectangle (5,2);}
%                    \onslide<2>{\draw[fill=OliveGreen] (5,2) rectangle (6,5);}
%                    \onslide<2>{\draw[fill=OliveGreen] (2,5) rectangle (5,6);}
%                    \onslide<2>{\draw[fill=OliveGreen] (1,2) rectangle (2,5);}
%
%                    \draw[step=1, help lines, dashed, color=black!30] (0,0) grid (7.5,7.5);
%                    \draw[->,thick,black] (0,0) -- (7.5,0) node [right] {$x_1$};
%                    \draw[->,thick,black] (0,0) -- (0,7.5) node [above] {$x_2$};
%
%                    %path
%                    \draw[very thick,black] (3,2) -- (4,2);
%                    \draw[very thick,black] (4,2) -- (5,2);
%                    \draw[very thick,black] (5,2) -- (5,3);
%                    \draw[very thick,black] (5,3) -- (5,4);
%                    \draw[very thick,black] (5,4) -- (5,5);
%                    \draw[very thick,black] (5,5) -- (4,5);
%                    \draw[very thick,black] (4,5) -- (3,5);
%                    \draw[very thick,black] (3,5) -- (2,5);
%                    \draw[very thick,black] (2,5) -- (2,4);
%                    \draw[very thick,black] (2,4) -- (2,3);
%                    \draw[very thick,black] (2,3) -- (2,2);
%                    \draw[very thick,black] (2,2) -- (3,2);
%
%                    %crosses
%                    %\onslide<2>{\draw (2,2) -- (3,1) -- (4,2) -- (5,1);}
%                    %\onslide<2>{\draw (2,1) -- (3,2) -- (4,1) -- (5,2);}
%
%                    %\onslide<2>{\draw (5,2) -- (6,3) -- (5,4) -- (6,5);}
%                    %\onslide<2>{\draw (6,2) -- (5,3) -- (6,4) -- (5,5);}
%                    %
%                    %\onslide<2>{\draw (2,6) -- (3,5) -- (4,6) -- (5,5);}
%                    %\onslide<2>{\draw (2,5) -- (3,6) -- (4,5) -- (5,6);}
%
%                    %\onslide<2>{\draw (1,2) -- (2,3) -- (1,4) -- (2,5);}
%                    %\onslide<2>{\draw (2,2) -- (1,3) -- (2,4) -- (1,5);}
%
%                \end{tikzpicture}
%                \end{overprint}
%            \end{center}
%        \end{column}
%    \end{columns}
%\end{frame}
%
%\begin{frame}
%    \frametitle{Algoritmo a cluster I}
%    \begin{columns}[T]
%    \centering
%        \begin{column}{0.59\textwidth}
%            \begin{itemize}
%                \item
%                    Se un link al bordo è reale, la plaquette esterna è invariata dopo l'inversione.
%                \item
%                    Eseguento una successione di trasformazioni di gauge locali, è possibile porre a 1 tutti i link del bordo tranne uno.
%                    Per esempio, per il percorso in figura:
%                    \begin{itemize}
%                        \item
%                            Si applica una trasformazione di gauge $G=U^*_{11}$ al sito di partenza di $U_{11}$, ponendolo a 1.
%                        \item
%                            Lo stesso procedimento si applica in successione su tutti i link $U_{10},U_{9},\ldots, U_1$.
%                    \end{itemize}
%            \end{itemize}
%        \end{column}
%        \begin{column}{0.39\textwidth}
%                \begin{tikzpicture}[x=0.35cm,y=0.35cm]
%                    %internal links
%                    %\draw[step=1,color=RoyalBlue] (1,1) grid (3,3);
%
%                    \onslide<1>{\draw[fill=MidnightBlue] (2,2) rectangle (5,5);}
%                    \onslide<2>{\draw[fill=Maroon] (2,2) rectangle (5,5);}
%                    %\draw[step=1,white] (2,2) grid (5,5);
%
%                    %external
%                    \onslide<2>{\draw[fill=OliveGreen] (2,1) rectangle (5,2);}
%                    \onslide<2>{\draw[fill=OliveGreen] (5,2) rectangle (6,5);}
%                    \onslide<2>{\draw[fill=OliveGreen] (2,5) rectangle (5,6);}
%                    \onslide<2>{\draw[fill=OliveGreen] (1,2) rectangle (2,5);}
%
%                    \draw[step=1, help lines, dashed, color=black!30] (0,0) grid (7.5,7.5);
%                    \draw[->,thick,black] (0,0) -- (7.5,0) node [right] {$x_1$};
%                    \draw[->,thick,black] (0,0) -- (0,7.5) node [above] {$x_2$};
%
%                    %path
%                    \draw[very thick,black] (3,2) -- (4,2);
%                    \draw[very thick,black] (4,2) -- (5,2);
%                    \draw[very thick,black] (5,2) -- (5,3);
%                    \draw[very thick,black] (5,3) -- (5,4);
%                    \draw[very thick,black] (5,4) -- (5,5);
%                    \draw[very thick,black] (5,5) -- (4,5);
%                    \draw[very thick,black] (4,5) -- (3,5);
%                    \draw[very thick,black] (3,5) -- (2,5);
%                    \draw[very thick,black] (2,5) -- (2,4);
%                    \draw[very thick,black] (2,4) -- (2,3);
%                    \draw[very thick,black] (2,3) -- (2,2);
%                    \draw[very thick,black] (2,2) -- (3,2);
%
%                    %crosses
%                    %\onslide<2>{\draw (2,2) -- (3,1) -- (4,2) -- (5,1);}
%                    %\onslide<2>{\draw (2,1) -- (3,2) -- (4,1) -- (5,2);}
%
%                    %\onslide<2>{\draw (5,2) -- (6,3) -- (5,4) -- (6,5);}
%                    %\onslide<2>{\draw (6,2) -- (5,3) -- (6,4) -- (5,5);}
%                    %
%                    %\onslide<2>{\draw (2,6) -- (3,5) -- (4,6) -- (5,5);}
%                    %\onslide<2>{\draw (2,5) -- (3,6) -- (4,5) -- (5,6);}
%
%                    %\onslide<2>{\draw (1,2) -- (2,3) -- (1,4) -- (2,5);}
%                    %\onslide<2>{\draw (2,2) -- (1,3) -- (2,4) -- (1,5);}
%
%                \end{tikzpicture}
%        \end{column}
%    \end{columns}
%\end{frame}
%
%\begin{frame}
%    \frametitle{Algoritmo a cluster II}
%    \begin{columns}[T]
%    \centering
%        \begin{column}{0.59\textwidth}
%            \begin{itemize}
%                \item
%                    Coniugando $U$ e tutti i link interni, solo la plaquette esterna $\mathcal P$ viene modificata $\Rightarrow$ accettanza non bassa: $10\%-30\%$.
%                \item
%                    L'algoritmo è stato ulteriormente ottimizzato con un'opportuna scelta gaussiana di $U$,
%                    ottenendo un'accettanza del $20\%-40\%$.
%                \item
%                    L'accettanza è indipendente dalla grandezza del cluster, che può essere scelta opportunamente visualizzando la carica.
%            \end{itemize}
%        \end{column}
%        \begin{column}{0.39\textwidth}
%        \begin{tikzpicture}[x=0.75cm,y=0.75cm]
%            \draw[step=1, help lines, dashed, color=black!30] (0,0) grid (4.5,4.5);
%            \draw[->,thick,black] (0,0) -- (4.5,0) node [right] {$x_1$};
%            \draw[->,thick,black] (0,0) -- (0,4.5) node [above] {$x_2$};
%
%            %gate
%            \draw[->-,very thick,black,dashed] (2,1) -- node [black,above] {$U$} (3,1);
%
%            %path
%            \draw[->-,very thick,RoyalBlue] (3,1) -- node [black,below] {$U_1$} (4,1);
%            \draw[->-,very thick,RoyalBlue] (4,1) -- node [black,right] {$U_2$} (4,2);
%            \draw[->-,very thick,RoyalBlue] (4,2) -- node [black,right] {$U_3$} (4,3);
%            \draw[->-,very thick,RoyalBlue] (4,3) -- node [black,right] {$U_4$} (4,4);
%            \draw[->-,very thick,RoyalBlue] (4,4) -- node [black,above] {$U_5$} (3,4);
%            \draw[->-,very thick,RoyalBlue] (3,4) -- node [black,above] {$U_6$} (2,4);
%            \draw[->-,very thick,RoyalBlue] (2,4) -- node [black,above] {$U_7$} (1,4);
%            \draw[->-,very thick,RoyalBlue] (1,4) -- node [black,left] {$U_8$} (1,3);
%            \draw[->-,very thick,RoyalBlue] (1,3) -- node [black,left] {$U_9$} (1,2);
%            \draw[->-,very thick,RoyalBlue] (1,2) -- node [black,left] {$U_{10}$} (1,1);
%            \draw[->-,very thick,RoyalBlue] (1,1) -- node [black,below] {$U_{11}$} (2,1);
%
%            %plaquette
%            \draw[->-,very thick,Maroon] (2,0) -- (2,1);
%            \draw[->-,very thick,Maroon] (3,0) -- (2,0);
%            \draw[->-,very thick,Maroon] (3,1) -- (3,0);
%            \node at (2.5,0.5) {$\mathcal P$};
%
%            %\draw[->-,very thick,dashed] (2,1) -- (3,1);
%        \end{tikzpicture}
%        \end{column}
%    \end{columns}
%\end{frame}
%
%\begin{frame}
%    \frametitle{Limite al continuo dell'accettanza dell'algoritmo a cluster: Inversione vs Gauss}
%    \centering
%    \import{gfx/}{cluster_acc_cont.pgf}
%\end{frame}
%
%\begin{frame}
%    \frametitle{Accettanza al variare della dimensione del cluster: Inversione vs Gauss}
%    \centering
%    \import{gfx/}{cluster_acc_side.pgf}
%\end{frame}
%
%\begin{frame}
%    \frametitle{Algoritmo a cluster: visualizzazione}
%    \begin{columns}[T]
%        \centering
%        \begin{column}{0.39\textwidth}
%            \begin{itemize}
%                \item
%                    L'algoritmo a cluster è applicato alla regione evidenziata.
%                \item
%                    La proposta è stata accettata e il cluster è stato invertito.
%                \item
%                    La colormap della carica mostra l'inversione di segno della carica all'interno della regione.
%                \item
%                    Anche il piccolo avvolgimento interno ha cambiato segno.
%            \end{itemize}
%        \end{column}
%        \begin{column}{0.59\textwidth}
%            \begin{overprint}
%                \onslide<1>\centerline{\import{gfx/}{cluster_inv1.pgf}}
%                \onslide<2>\centerline{\import{gfx/}{cluster_inv2.pgf}}
%            \end{overprint}
%        \end{column}
%    \end{columns}
%\end{frame}
%
%\begin{frame}
%    \frametitle{Storia della carica topologica: locale vs cluster}
%    \centering
%    \import{gfx/}{cluster_charge_history.pgf}
%\end{frame}
%
%
%\begin{frame}
%    \frametitle{Algoritmo a cluster: simulazioni e verifica della correttezza}
%    \begin{itemize}
%        \item
%            Sono state effettuate simulazioni con l'algoritmo a cluster sulla stessa linea di volume fisico costante $\beta/N^2=1/80$ e
%            con lo stesso numero di iterazioni usate per l'algoritmo locale.
%        \item
%            In questo caso le iterazioni sono un'alternanza di aggiornamenti locali e di inversioni di cluster.
%        \item
%            Per verificare la correttezza dell'algoritmo, è stato nuovamente confrontato il valore di $E[\mathcal P]$ con $\beta=7.2$ e $N=24$:
%            \begin{equation*}\begin{aligned}
%                E[\mathcal P] &= 0.52040(39) \quad \text{(D\"urr-Hoelbling)} \\
%                E[\mathcal P] &= \input{tables/plaq_local.tex} \quad \text{(Algoritmo locale)} \\
%                E[\mathcal P] &= \input{tables/plaq_cluster.tex} \quad \text{(Algoritmo a cluster)}
%            \end{aligned}\end{equation*}
%    \end{itemize}
%\end{frame}
%
%\begin{frame}
%    \frametitle{Algoritmo a cluster: limite al continuo dell'energia di plaquette media}
%    \centering
%    \import{gfx/}{cluster_energy_cont.pgf}
%\end{frame}
%
%\begin{frame}
%    \frametitle{Algoritmo a cluster: limite al continuo della suscettività topologica I}
%    \centering
%    \import{gfx/}{cluster_susc_cont.pgf}
%\end{frame}
%
%\begin{frame}
%    \frametitle{Algoritmo a cluster: limite al continuo della suscettività topologica II}
%    \begin{itemize}
%        \item
%            Risultati dei fit:
%            \input{tables/cluster_susc_cont.tex}
%        \item
%            L'andamento verso il limite al continuo cambia per reticoli sufficientemente fini,
%            quando è più marcata la struttura a cluster.
%        \item
%            Questo cambio di regime non è apprezzabile con misure più imprecise.
%        \item
%            Non è stato osservato un effetto sistematico includendo o escludendo dal fit il punto a $\beta=5$.
%    \end{itemize}
%\end{frame}
%
%\section{Conclusioni e prospettive}
%
%\begin{frame}
%    \frametitle{Conclusioni}
%    \begin{itemize}
%        \item
%            È stato definito un originale algoritmo M-H locale per le teorie $U(1)$,
%            che sfrutta il metodo di Laplace per ottenere accettanze molto alte vicino al limite continuo.
%        \item
%            È stato introdotto un originale metodo di smoothing della carica basato sul blur Gaussiano noto in image processing.
%            Questo metodo si è rivelato utile per studiare le configurazioni topologiche del sistema.
%        \item
%            È stato definito un originale algoritmo M-H a cluster che facilita il tunneling della carica contrastando efficacemente il freezing topologico.
%            L'algoritmo è stato ulteriormente ottimizzato con una scelta Gaussiana in modo analogo al M-H locale.
%        \item
%            È stata effettuata un'estrapolazione della suscettività topologica ad alta precisione:
%            distinguendo due diversi regimi è stato possibile  migliorare l'estrapolazione di D\"urr e Hoelbling.
%    \end{itemize}
%\end{frame}
%
%\begin{frame}
%    \frametitle{Prospettive future}
%    \begin{itemize}
%        \item
%            I risultati esatti forniti da questo algoritmo possono essere usati per valutare le comuni tecniche applicate in QCD per contrastare il freezing topologico.
%        \item
%            È possibile introdurre dei fermioni in questo sistema e valutare le performance dell'algoritmo a cluster in loro presenza.
%            Sono particolarmente interessanti perché confinati in modo simile a quanto avviene in QCD.
%        \item
%            Visti gli ottimi risultati ottenuti in questo modello,
%            sono in corso dei tentativi di estensione dell'algoritmo a cluster a modelli più complessi, come il $\mathbb CP^{N-1}$ e la QCD.
%    \end{itemize}
%\end{frame}
%
%\hoffset=-0.5\leftsidebar
%\begin{frame}[plain,t]
%\titlepage
%\end{frame}
%\hoffset=0in % restore left margin
%
%\section{Appendice}
%
%\begin{frame}
%    \frametitle{Metodo di Laplace}
%    \centering
%    \import{gfx/}{gauss.pgf}
%\end{frame}
%
%\begin{frame}
%    \frametitle{Gaussiana troncata}
%    \begin{equation*}\begin{gathered}
%        x = \sqrt{-\frac{2}{k}\log\left[1-y_1\left(1-e^{-\frac{k}{2}\pi^2}\right)\right]}\cos\left[2\pi\left(y_2-\frac{1}{2}\right)\right] \\[1em]
%        \text{con:}\ 
%        \begin{dcases}
%             y_1 \in [0,1] \\
%             y_2 \in (0,1)
%        \end{dcases}
%    \end{gathered}\end{equation*}
%    \[
%        p(x) \propto e^{-kx^2/2} \qquad x\in (-\pi,\pi]
%    \]
%\end{frame}
%
%\begin{frame}
%    \frametitle{Algoritmo locale: storia dell'energia di plaquette media}
%    \centering
%    \import{gfx/}{local_energy_history.pgf}
%\end{frame}
%
%\begin{frame}
%    \frametitle{Limite al continuo della suscettività topologica con BIAS}
%    \centering
%    \import{gfx/}{local_susc_cont.pgf}
%\end{frame}
%
%\begin{frame}
%    \frametitle{Blur Gaussiano}
%	\begin{overprint}
%		\onslide<1>\centerline{\import{gfx/}{cappadocia1.pgf}}
%		\onslide<2>\centerline{\import{gfx/}{cappadocia2.pgf}}
%		\onslide<3>\centerline{\import{gfx/}{cappadocia3.pgf}}
%		\onslide<4>\centerline{\import{gfx/}{cappadocia4.pgf}}
%	\end{overprint}
%\end{frame}
%
%
%\begin{frame}
%    \frametitle{Algoritmo a cluster: tempo di autocorrelazione della carica topologica}
%    \centering
%    \import{gfx/}{cluster_charge_corr.pgf}
%\end{frame}

\end{document}
